\documentclass[12pt, a4paper]{report}
\usepackage{graphicx}
\usepackage[utf8]{inputenc}
\usepackage[T1]{fontenc}
\usepackage[top=1in, bottom=1in, left=1in, right=1in]{geometry}
\author{Grupp 2}
\title{Planering AU2}

\begin{document}

\begin{figure}[t]

\includegraphics[scale=1]{mah}
\end{figure}

\begin{flushleft}
Malmö högskola\\
Fakulteten för teknik och samhälle
\end{flushleft}


\begin{center}
\begin{Huge}

Plan för arbetsuppgift 2:

\end{Huge}
\begin{LARGE}

Resonans

\end{LARGE}
\end{center}
\begin{flushleft}
\begin{large}
Grupp 2:\\
\end{large}

David Tran\\
John Tengvall\\
Jonathan Böcker\\
Karl-Andreas Langhammer Fielitz\newline

\textbf{\textit{
Beskriv, med egna ord, vad arbetsuppgiften går ut på.
}}

Arbetsuppgiften handlar om begreppet resonans.
I arbetsuppgiften går vi igenom vad resonans är, hur och vart det kan uppstå och hur man med hjälp av matematik kan räkna ut förutsättningarna för när resonans uppstår. 
Begreppet differentialekvationer utgör en stor del av uppgiften eftersom det behövs för att lösa den bakomliggande matematiken i systemet. \newline

I denna uppgiften ska vi även dra paralleller till ett massa-fjäder-dämpare system. Detta kommer vara av stor vikt då vi ska kunna förstå vilka fysikaliska storheter samt hur energi fördelas i vår RLC-krets. Då kommer MFD-systemet vara av hjälp då detta mekaniska system liknar vår RLC-krets i flera fall.\newline

En del är, som nämnt ovan, att förstå hur en RLC-krets fungerar. Med hjälp av resistans, induktans och kapacitans kan det uppstå resonans i kretsen och vi ska räkna ut hur mycket med hjälp av olika värden på komponenterna. En RLC-krets kan fungera som ett analogt bandpass-filter och med hjälp av Matlab kan vi plotta ut frekvensresponsen.\newline

\textbf{\textit{
Redogör för hur ni kommer att lösa arbetsuppgiften inom given tid och ange vem som gör vad.
}}

Mycket i denna arbetsuppgift är nytt för gruppen. Begrepp som resonans och impedans är bekant, men utan djupare förståelse så får gruppen ta stor hjälp av Physics boken. I boken tar Walker upp resonans och hur det uppstår i RLC krets.\newline

Differentialekvationer är heller inte vida känt i gruppen så där får vi gå igenom vad det är, hur man räknar ut ekvationerna, och inom vilka områden i verkliga livet dess ekvationer tillämpas.\newline

Vi kommer arbeta över Google-drive med de kompletterande frågorna samt själva rapporten. Vi har delat upp frågorna inom gruppen, tanken är sen att varje person kommer informera de andra med en liten föreläsning så att alla förstår frågorna och uppgiften i helhet. John och David kommer att ta sig an uppgifterna om differentialekvationer. Jonathan och Andreas svarar på frågorna om RLC-krets. De sista två delarna, mekaniskt system och resonans, kommer Andreas och David att lösa.  Vi kommer slutligen använda oss av LaTeX för att optimera rapporten innan vi skickar in vårt slutliga resultat.\newline

Vi har valt att arbeta tillsammans med simulink och Matlab då vi fortfarande inte är helt hundra på att koda med den syntaxen. Större delen av uppgift blir att generera grafer, plottar och uträkningar med programmen så för att alla ska förstå djupet av koden behöver varje person i gruppen vara delaktig. 

\end{flushleft}
\end{document}
